\documentclass[10pt, a4paper]{article}
\usepackage[paper=a4paper, left=1.5cm, right=1.5cm, bottom=1.5cm, top=3.5cm]{geometry}
\usepackage[utf8]{inputenc}
\usepackage[T1]{fontenc}
\usepackage[spanish]{babel}
\usepackage{indentfirst}
\usepackage{fancyhdr}
\usepackage{latexsym}
\usepackage{lastpage}
\usepackage{aed2-symb,aed2-itef,aed2-tad}
\usepackage{caratula}
\usepackage[colorlinks=true, linkcolor=blue]{hyperref}
\usepackage{calc}

\newcommand{\f}[1]{\text{#1}}
\renewcommand{\paratodo}[2]{$\forall~#2$: #1}

\sloppy

\hypersetup{%
 % Para que el PDF se abra a p�gina completa.
 pdfstartview= {FitH \hypercalcbp{\paperheight-\topmargin-1in-\headheight}},
 pdfauthor={Algoritmos y Estructuras de Datos II - DC - UBA}
% pdfkeywords={TADs b�sicos},
% pdfsubject={Tipos abstractos de datos b�sicos}
}

\parskip=5pt % 10pt es el tama�o de fuente

% Pongo en 0 la distancia extra entre �temes.
\let\olditemize\itemize
\def\itemize{\olditemize\itemsep=0pt}

% Acomodo fancyhdr.
\pagestyle{fancy}
\thispagestyle{fancy}
\addtolength{\headheight}{1pt}
\lhead{Algoritmos y Estructuras de Datos II}
\rhead{$1^{\mathrm{er}}$ cuatrimestre de 2016}
\cfoot{\thepage /\pageref{LastPage}}
\renewcommand{\footrulewidth}{0.4pt}

%\author{Algoritmos y Estructuras de Datos II, DC, UBA.}
%\date{}
%\title{Tipos abstractos de datos b�sicos}

\begin{document}

\titulo{Trabajo Practico 1: "Base de Datos"}
\materia{Algoritmos y Estructuras de Datos II, DC, UBA.}
\integrante {Gabriel Salvo}{564/14}{gabrielsalvo.cap@gmail.com}
\integrante {German Ariel Cuacci}{609/14}{germancuacci@gmail.com}
\integrante {Joel Esteban Camera}{257/14}{joel.e.camera@gmail.com}
\integrante {Martin Jonas}{180/05}{martinjonas@gmail.com}


%Pagina de titulo e indice
\thispagestyle{empty}

\maketitle
\tableofcontents


\newpage


\section{Renombres de TADs}


\tadNombre{TAD TipoDeCampo} es \tadNombre{Enum \{NAT, STRING\}}

\tadNombre{TAD Campo} es \tadNombre{String}

\tadNombre{TAD Campos} es \tadNombre{DiccExt(Campo, TipoDeCampo)}

\tadNombre{TAD ClavesT} es \tadNombre{Conjunto(Campo)}

\tadNombre{TAD Registro} es \tadNombre{DiccExt(Campo, Valor)}

\tadNombre{TAD Registros} es \tadNombre{Conjunto(Registro)}

\tadNombre{TAD tablasdb} es \tadNombre{DiccExt(String,Tabla)}

\tadNombre{TAD join} es \tadNombre{Tupla(t1: String, t2: String, clave: String)}

\tadNombre{TAD trigger} es \tadNombre{Tupla(t1: String, t2: String, r: Registro)}



\section{Notas}

\begin{itemize}

%Asumimos la existencia de la igualdad (no observacional) en todos los tipos basicos. \\

\item Al momento de devolver registros en verJoin, si la tabla 2 del join tiene campos con el mismo nombre que la tabla 1 ademas de la clave utilizada para reralizar el join, estos campos no se muestran en el join.

\item Al momento de agregar un registro en una tabla, si esta tabla tuviese un trigger, y el registro resultante a ser agregado en la tabla 2 no se pudiese agregar, el registro en la tabla 1 es agregado pero el registro desencadenado no.
	
\item Asumimos la existencia de la igualdad (no observacional) en todos los tipos basicos.

\item Definimos una seccion de  ``otros predicados'' para definir predicados auxiliares, para ser utilizados unicamente en las restricciones, y que por lo tanto permiten el uso de cuantificadores universales.

	
	
\end{itemize}
	
	

%TADS

\newpage
\section{TAD \tadNombre{DiccExt}}
\begin{tad}{\tadNombre{DiccExt}}
\tadGeneros{diccext}
\tadExporta{diccext, otras operaciones }
\tadUsa{ 
 \tadNombre{Conjunto(clave)}
 }
\tadExtiende{ 
 \tadNombre{Diccionario}
 }

\tadParametrosFormales{
    \tadEncabezadoInline{generos}{clave, significado}
}


\tadAlinearFunciones{borrarVarios}{conj(clave)/cs,dicc(clave, significado)/di}

\tadOtrasOperaciones

\tadOperacion{borrarVarios}{conj(clave)/cs,dicc(clave, significado)/di}{dicc(clave, significado)}{cs $\subseteq$ claves(di) }
\tadOperacion{borrarResto}{conj(clave)/cs,dicc(clave, significado)/di}{dicc(clave, significado)}{cs $\subseteq$ claves(di) }
\tadOperacion{unir}{dicc(clave, significado)/di,dicc(clave, significado)/dn}{dicc(clave, significado)}{ vacio?(claves(di) $\cap$ claves(dn)) }
\tadOperacion{primc}{dicc(clave, significado)/di}{clave}{ $\lnot$vacio?(di) }
\tadOperacion{sinc}{dicc(clave, significado)/di}{dicc(clave, significado)}{ $\lnot$vacio?(di) }
\tadOperacion{$\argumento$ [ $\argumento$ ]  }{dicc(clave, significado)/di,clave/a}{ significado}{ def?(a, di)  }


\tadAxiomas[\paratodo{conj(clave)}{cs}, \paratodo{dicc(clave,significado)}{di,dn}, \paratodo{clave}{a}]

\tadAlinearAxiomas{borrarVarios(cs,di)}


\tadAxioma{borrarVarios(cs,di)}{ \IF vacio?(cs) THEN di 
						ELSE  borrarVarios(
									sinUno(cs), borrar( dameUno(cs), di ) 
								) 
					 FI  }


\tadAxioma{borrarResto(cs,di)}{ borrarVarios(claves(di) - cs, di) }

\tadAxioma{unir(di,dn)}{ \IF vacio?(dn) THEN di 
									ELSE
										unir( 								
											definir( primc(dn), dn[primc(dn)], di) ), 
											sinc(dn)
										)
									FI }

\tadAxioma{primc(di)}{ dameUno(claves(di)) }
\tadAxioma{sinc(di)}{ borrar( primc(di), di)  }
\tadAxioma{di [ a ]  }{ obtener(a, di) }




\end{tad}

\newpage
\section{TAD \tadNombre{Valor}}
\begin{tad}{\tadNombre{Valor}}
\tadGeneros{valor}
\tadExporta{valor, generadores, observadores,  = }
\tadUsa{ 
 \tadNombre{String}, 
 \tadNombre{Nat}, 
 \tadNombre{TipoDeCampo}
 }


\tadIgualdadObservacional{x}{y}{valor}{
					 tipo(x) $\igobs$ tipo(y) $\yluego$ 
					 (tipo(x) $\igobs$ NAT $\yluego$ valorNat(x) $\igobs$ valorNat(y)) $\oluego$
					 (tipo(x) $\igobs$ STRING $\yluego$ valorString(x) $\igobs$ valorString(y))
}

\tadAlinearFunciones{valorDeString}{valor/v,valor/v2}

\tadObservadores
\tadOperacion{tipo}{valor}{tipodecampo}{}
\tadOperacion{valorDeNat}{valor/v}{nat}{tipo(v) = NAT}
\tadOperacion{valorDeString}{valor/v}{string}{tipo(v) = STRING}

\tadGeneradores
\tadOperacion{valorNat}{nat}{valor}{}
\tadOperacion{valorString}{string}{valor}{}


\tadOtrasOperaciones
\tadOperacion{$\argumento$  = $\argumento$ }{valor/v,valor/v'}{bool}{}



\tadAxiomas[\paratodo{v,v'}{valor}, \paratodo{string}{s}, \paratodo{nat}{n}]

\tadAlinearAxiomas{valorDeString(valorString(s))}
\tadAxioma{tipo(valorNat(s))}{NAT}
\tadAxioma{tipo(valorString(s))}{STRING}
\tadAxioma{valorDeNat(valorNat(n))}{n}
\tadAxioma{valorDeString(valorString(s))}{s}

\tadAxioma{v = v'}{ tipo(v) = tipo(v') $\yluego$ 
	\IF tipo(v) = NAT THEN 
		valorDeNat(v) = valorDeNat(v') 
	ELSE 
		valorDeString(v) = valorDeString(v') 
	FI  
}




\end{tad}

\newpage
\section{TAD \tadNombre{Tabla}}


\begin{tad}{\tadNombre{Tabla}}
\tadGeneros{tabla}
\tadExporta{tabla, generadores, observadores, puedoAgregar?, puedoBorrar?, verJoin }
\tadUsa{\tadNombre{Campos}, \tadNombre{Claves}, \tadNombre{Registros}, \tadNombre{Nat}, \tadNombre{Bool} }


%===============================================
%   IGUALDAD OBS
%===============================================

\tadIgualdadObservacional{t}{s}{tabla}{
		 camposDe(t) $\igobs$ camposDe(s) $\land$
		 clavesDe(t) $\igobs$ clavesDe(s) $\land$
		 registrosDe(t) $\igobs$ registrosDe(s) $\land$
		 \#cambios(t) $\igobs$ \#cambios(s)
}


%===============================================
%   OBSERVADORES
%===============================================

\tadAlinearFunciones{cantidadValores}{registro/r,campo/c,valor/v}

\tadObservadores
\tadOperacion{camposDe}{tabla}{campos}{}
\tadOperacion{clavesDe}{tabla}{clavest}{}
\tadOperacion{registrosDe}{tabla}{registros}{}
\tadOperacion{\#cambios}{tabla}{nat}{}


%===============================================
%   GENERADORES
%===============================================

\tadGeneradores
\tadOperacion{nuevaTabla}{campos/camp,claves/cl}{tabla}
{ $\lnot$ vacio?(claves(camp)) $\land$ $\lnot$ vacio?(cl) $\land$ cl $\subseteq$ claves(camp)  }
\tadOperacion{agregarRegistro}{tabla/t,registro/r}{tabla}{ 
	regValido?(t,r) $\yluego$
	puedoAgregar?(t,r) }
\tadOperacion{borrarRegistros}{tabla/t,clave/c,valor/v}{tabla}{ cvValido(t,c,v) $\yluego$  puedoBorrar?(t,c,v) }




%===============================================
%   OTRS PREDICADOS
%===============================================
\textbf{otros predicados}

\tadOperacion{regValido?}{tabla/t,registro/r}{bool}{ 
	claves(camposDe(t)) = claves(r) $\yluego$  
	$(\forall s:\text{string})$ def?(s,r) $\impluego$ tipo(r[s]) = camposDe(t)[s] }

\tadOperacion{cvValido?}{tabla/t,clave/c,valor/v}{bool}{ 
	def?(c, camposDe(t)) $\yluego$ camposDe(t)[c] = tipo(v) }


%===============================================
%   OTRAS OPERACIONES
%===============================================

\tadOtrasOperaciones

\tadOperacion{puedoAgregar?}{tabla/t,registro/r}{bool}{ regValido?(t,r) }
\tadOperacion{puedoBorrar?}{tabla/t,clave/c,valor/v}{bool}{ cvValido?(t,c,v) }

\tadOperacion{regenerar}{tabla/t,clave/c,valor/v}{tabla}{}


\tadOperacion{todosDistintos}{registro/r1,registro/r2}{bool}{}
\tadOperacion{soloCampClaves}{tabla/t,registro/r}{registro}{}


\tadOperacion{verJoin}{tabla/t1,tabla/t2,campo/c}{registros}{
c $\in$ (clavesDe(t1) $\cap$ clavesDe(t2)
$\yluego$ \\camposDe(t1)[c] = camposDe(t2)[c] 
}
\tadOperacion{verJoinAux}{tabla/t1,tabla/t2,registros/rs,campo/c}{registros}{}
\tadOperacion{regJoin}{tabla/t2,registros/rs,campo/c}{registro}{}
\tadOperacion{tieneRegistro?}{tabla/t2,campo/c,valor/v}{bool}{}
\tadOperacion{obtenerRegistro}{registros/rs,campo/c,valor/v}{registro}{}


%===============================================
%   AXIOMAS
%===============================================

\newpage

\tadAxiomas[\paratodo{registro}{r,r1,r2}, \paratodo{campo}{c,c'},\paratodo{campos}{camp}, \paratodo{clavest}{cl,clr1,clr2}, \paratodo{valor}{v,v'}]

\tadAlinearAxiomas{puedoAgregarbla(nuevaTabla(camp,cl),c,v)}



%-----------------------
%   ax: observadores 
%-----------------------

\tadAxioma{camposDe(nuevaTabla(camp,cl))}{camp}
\tadAxioma{camposDe(agregarRegistro(t,r))}{camposDe(t)}
\tadAxioma{camposDe(borrarRegistros(t,c,v))}{camposDe(t)}

\tadAxioma{clavesDe(nuevaTabla(camp,cl))}{cl}
\tadAxioma{clavesDe(agregarRegistro(t,r))}{clavesDe(t)}
\tadAxioma{clavesDe(borrarRegistros(t,c,v))}{clavesDe(t)}


\tadAxioma{registrosDe(nuevaTabla(camp,cl))}{$\emptyset$ }
\tadAxioma{registrosDe(agregarRegistro(t,r))}{Ag(r, registrosDe(t))}
\tadAxioma{registrosDe(borrarRegistros(t,c,v))}{ registrosDe(regenerar(t,c,v))  }

\tadAxioma{\#cambios(nuevaTabla(camp,cl))}{0}
\tadAxioma{\#cambios(agregarRegistro(t,r))}{ 1 + \#cambios(t) }
\tadAxioma{\#cambios(borrarRegistros(t,c,v))}{ \#(registrosDe(t)) -\#(registrosDe(borrarRegistros(t,c,v)) + \#cambios(t)  }


%-----------------------
%   ax: restricciones 
%-----------------------

\tadAxioma{puedoAgregar?(nuevaTabla(camp,cl),r)}{ true  }
\tadAxioma{puedoAgregar?(agregarRegistro(t,r'),r)}{ todosDistintos(soloCamposClave(t,r),  soloCamposClave(t,r')) $\land$ puedoAgregar?(t,r) }
\tadAxioma{puedoAgregar?(borrarRegistros(t,c,v),r)}{ puedoAgregar?(regenerar(t,c,v),r)  }

\tadAxioma{puedoBorrar?(nuevaTabla(camp,cl),c,v)}{ false  }
\tadAxioma{puedoBorrar?(agregarRegistro(t,r),c,v)}{ obtener(c,r)=v $\lor$ puedoBorrar?(t,c,v)  }
\tadAxioma{puedoBorrar?(borrarRegistros(t,c',v'),c,v)}{ puedoBorrar?(regenerar(t,c',v'),c,v)  }


%-----------------------
%   ax: regenerar 
%-----------------------

\tadAxioma{regenerar(nuevaTabla(camp,cl),c,v)}{nuevaTabla(camp,cl)}
\tadAxioma{regenerar(agregarRegistro(t,r),c,v)}{ \IF\ obtener(c,r)=v THEN regenerar(t,c,v) ELSE agregarRegistro(regenerar(t,c,v),r) FI }
\tadAxioma{regenerar(borrarRegistros(t,c',v'),c,v)}{ borrarRegistros(regenerar(t,c,v),c',v') }

\tadAxioma{todosDistintos(r1,r2)}{ \IF vacio?(claves(r1)) THEN true ELSE { 
				\IF r1[primc(r1)] = r2[primc(r1)] THEN false ELSE
					todosDistintos(sinc(r1), r2)
				FI
			} FI
  }
\tadAxioma{soloCampClaves(t,r)}{ borrarResto( clavesDe(t), r)  }




%-----------------------
%   ax: verJoin
%-----------------------

\tadAxioma{verJoin(t1,t2,rs,c)}{ verJoinAux(t1,t2,registrosDe(t1),c) }

\tadAxioma{verJoinAux(t1,t2,rs,c)}{
	\IF vacio?(rs) THEN $\emptyset$ ELSE { 
		\IF tieneRegistro?(t2, c, dameUno(rs)[c]) THEN
			Ag( regJoin(t2, rs, c), verJoinAux(t1,t2,sinUno(rs),c) )
		ELSE
			verJoinAux(t1,t2,sinUno(rs),c)
		FI }
	FI
}

\tadAxioma{regJoin(t2,rs,c)}{  unir(  obtenerRegistro( registrosDe(t2), c, dameUno(rs)[c] ),   dameUno(rs)  )  }
\tadAxioma{tieneRegistro?(t2,c,v)}{  puedoBorrar?(t2, c, v)  }
\tadAxioma{obtenerRegistro(rs, c, v)}{
	\IF dameUno(rs)[c] = v THEN dameUno(rs) ELSE
		obtenerRegistro(sinUno(rs),c,v)
	FI	
}




\end{tad}


\newpage
\section{TAD \tadNombre{Basededatos}}



\begin{tad}{\tadNombre{Basededatos}}
\tadGeneros{bd}
\tadExporta{bd, generadores, observadores, eliminarJoin, eliminarTrigger, verJoin, tsMasModificadas}
\tadUsa{\tadNombre{tabla}, \tadNombre{valor}}


%===============================================
%   IGUALDAD OBS
%===============================================
\tadIgualdadObservacional{b1}{b2}{bd}{
	 joins(b1) \igobs joins(b2) $\land$ \\
	 (tablas(b1) \igobs tablas(b2) $\yluego$ \\
	 $(\forall s:\text{string})$ s $\in$ claves(tablas(b1)) $\impluego$ triggers(b1,s) \igobs triggers(b2,s) )
}



%===============================================
%   OBSERVADORES
%===============================================
\tadAlinearFunciones{igualTipoCamposComunes}{bd/b,string/t,campo/c,valor/v}

\tadObservadores
\tadOperacion{tablas}{bd}{tablasbd}{}
\tadOperacion{joins}{bd}{conj(join)}{}
%puede que tenga sentido tomar una tabla t para triggers porque los triggers son de una tabla conceptualmente
%y en la axiomatizacion no cambia nada
\tadOperacion{triggers}{bd/b,string/n}{conj(trigger)}{ def?(n, tablas(b)) }


%===============================================
%   GENERADORES
%===============================================
\tadGeneradores
\tadOperacion{crearBasededatos}{tablasbd}{bd}{}

\tadOperacion{agregarRegistro}{bd/b,string/n,registro/r}{bd}{def?(n, tablas(b)) $\yluego$ regValido?(tablas(b)[n], r)  }

\tadOperacion{eliminarRegistros}{bd/b,string/n,campo/c,valor/v}{bd}{def?(n, tablas(b)) $\yluego$ cvValido?(tablas(b)[n],c,v) } 

\tadOperacion{agregarJoin}{bd/b,join/j}{bd}{(def?(j.t1, tablas(b)) $\land$ def?(j.t2, tablas(b))) $\yluego$ joinValido?(tablas(b)[j.t1], tablas(b)[j.t2]), j) }

\tadOperacion{agregarTrigger}{bd/b,trigger/tr}{bd}{ def?(tr.t1, tablas(b)) $\land$ def?(tr.t2, tablas(b)) $\yluego$ trigValido?(tablas(bd)[tr.t1], tablas(bd)[tr.t2], tr.r) }



%===============================================
%   OTRS PREDICADOS
%===============================================
\textbf{otros predicados}

\tadOperacion{regValido?}{tabla/t,registro/r}{bool}{ 
	claves(camposDe(t)) = claves(r) $\yluego$  
	$(\forall s:\text{string})$ def?(s,r) $\impluego$ tipo(r[s]) = camposDe(t)[s] $\yluego$ 
	puedoAgregar(t,r)	
	 }

\tadOperacion{cvValido?}{tabla/t,clave/c,valor/v}{bool}{ 
	def?(c, camposDe(t)) $\yluego$ camposDe(t)[c] = tipo(v) $\yluego$ puedoBorrar?(t, c, v)}

\tadOperacion{joinValido?}{tabla/t1,tabla/t2,join/j}{bool}{ 
 j.clave $\in$ (clavesDe(t1) $\cap$ clavesDe(t2)
$\yluego$ \\camposDe(t1)[j.clave] = camposDe(t2)[j.clave] 
}


\tadOperacion{trigValido?}{tabla/t1,tabla/t2,registro/r}{bool}{ 
	clavesDe(t1) $\subseteq$ clavesDe(t2) $\yluego$ \\
	igualCampos?( camposDe(t1), camposDe(t2) ) $\yluego$ \\
	claves(r) $\subseteq$ claves(camposDe(t2))  $\yluego$ \\
	igualTipoReg?(camposDe(t2), r)	
}

\tadOperacion{igualCampos?}{ campos/cs1,campos/cs2}{bool}{ 
	$(\forall c:\text{campo})$ ( def?(c, cs1) $\land$ def?(c, cs2) ) $\impluego$ cs1[c] = cs2[c]
}

\tadOperacion{igualTipoReg?}{ campos/cs,registro/r}{bool}{ 
	$(\forall c:\text{campo})$ def?(c, r) $\impluego$ (def?(c,cs) $\yluego$ tipo(r[c]) = cs[c])
}



%===============================================
%   OTRAS OPERACIONES
%===============================================

\newpage

\tadOtrasOperaciones

\tadOperacion{eliminarJoin}{bd/b,join/j}{bd}{j $\in$ joins(b)}

\tadOperacion{eliminarTrigger}{bd/b,trigger/t}{bd}{t $\in$ triggers(b)}
\tadOperacion{regenerarBD}{bd/b,conj(join)/js,conj(trigger)/trs}{bd}{}


\tadOperacion{verJoin}{bd/b,join/j}{registros}{j $\in$ joins(b)}

\tadOperacion{aplicarTriggers}{conj(trigger)/trs,registro/r,tablasbd/ts }{tablasbd}{}
\tadOperacion{aplicarTrigger}{trigger/tr,registro/r,tablasbd/ts }{tablasbd}{}
\tadOperacion{regDeTrig}{trigger/tr,registro/r,tablasbd/ts}{registro}{}

\tadOperacion{camposSoloT1}{tabla/t1,tabla/t2}{conj(campo)}{}
\tadOperacion{camposComunes}{tabla/t1,tabla/t2}{conj(campo)}{}


\tadOperacion{tsMasModificadas}{bd/b}{tablasbd}{}
\tadOperacion{masModificadas}{conj(string)/cs,tablasbd/ts,nat/m}{tablasbd}{}
\tadOperacion{maxCambios}{tablasbd/ts}{nat}{}


%===============================================
%  AXIOMATIZACION
%===============================================

\tadAxiomas[\paratodo{bd}{b},\paratodo{tabla}{t,t1,t2},\paratodo{tablasbd}{ts},\paratodo{join}{j},\paratodo{trigger}{tr},\paratodo{conj(trigger)}{trs},\paratodo{registro}{r},\paratodo{string}{n},\paratodo{conj(join)}{js},\paratodo{conj(string)}{cs},\paratodo{nat}{m}]

\tadAlinearAxiomas{regenerar(agregnueTablabla(camp,cl))}

%-----------------------
%   ax: obs
%-----------------------

\tadAxioma{tablas(crearBasededatos(ts))}{ts}
\tadAxioma{tablas(agregarRegistro(b,n,r))}{
			aplicarTriggers(triggers(b, n), r, definir(n, agregarRegistro(tablas(b)[n], r),  tablas(b)	))
}
\tadAxioma{tablas(eliminarRegistros(b,n,c,v))}{
			definir(n, eliminarRegistros(tablas(b)[n], c, v),  tablas(b)	))
}

\tadAxioma{tablas(agregarTrigger(b,tr))}{tablas(b)}

\tadAxioma{joins(crearBasededatos(t))}{$\emptyset$}
\tadAxioma{joins(agregarRegistro(b,n,r))}{joins(b)}
\tadAxioma{joins(eliminarRegistros(b,n,c,v))}{joins(b)}
\tadAxioma{joins(agregarJoin(b,j))}{Ag(joins(b),j)}
\tadAxioma{joins(agregarTrigger(b,tr))}{joins(b)}

\tadAxioma{triggers(crearBasededatos(t), n)}{$\emptyset$}
\tadAxioma{triggers(agregarRegistro(b,n,r), n)}{triggers(b,n)}
\tadAxioma{triggers(eliminarRegistros(b,n,c,v), n)}{triggers(b,n)}
\tadAxioma{triggers(agregarJoin(b,j), n)}{triggers(b,n)}
\tadAxioma{triggers(agregarTrigger(b,tr), n )}{ \IF tr.t1 $=$ n THEN Ag( triggers(b,n), tr)
	ELSE  triggers(b,n)
	FI
}



%-----------------------
%   ax: aplicaTrigger
%-----------------------

\tadAxioma{aplicarTriggers(trs,r,ts)}{ \IF vacio?(trs) THEN ts 
	ELSE 
		aplicarTriggers( sinUno(trs), r,  aplicarTrigger( dameUno(trs), r, ts ) )
	FI
}

\tadAxioma{aplicarTrigger(tr,r,ts)}{ \IF puedoAgregar?(ts[tr.t2], regDeTrig(tr, r, ts))
	THEN definir(tr.t2, agregarRegistro(ts[tr.t2],  regDeTrig(tr, r, ts)), ts)
	ELSE ts
	FI
 }

\tadAxioma{regDeTrig(tr, r, ts)}{  unir( borrarVarios(camposSoloT1(ts[tr.t1], ts[tr.t2]), r), tr.r) }

\tadAxioma{camposSoloT1(t1, t2)}{   claves(camposDe(t1)) - camposComunes(t1,t2)  }

\tadAxioma{camposComunes(t1,t2)}{ claves(camposDe(t1)) $\cap$ claves(camposDe(t2)) }


%-----------------------
%   ax: eliminarJoin eliminarTrigger opcion 1
%-----------------------


\tadAxioma{eliminarJoin(b,j)}{regenerarBD( crearBasededatos( tablas(b)), joins(b) - \{j\}, triggers(b) )}
\tadAxioma{eliminarTrigger(b,t)}{regenerarBD( crearBasededatos( tablas(b)), joins(b), triggers(b) - \{t\} )}

\tadAxioma{regenerarBD(b,js,trs)}{\IF vacio?(js) $\land$ vacio?(trs) THEN b
              ELSE {
                    \IF vacio?(js) THEN
                           regenerarBD(agregarTrigger(b,dameUno(trs)),js,sinUno(trs))
                     ELSE 
                           regenerarBD(agregarJoin(b,dameUno(js)),sinUno(js),trs)
                     FI
              }
              FI
}

%-----------------------
%   ax: verJoin opcion 
%-----------------------

\tadAxioma{verJoin(b,j)}{verJoin(tablas(b)[j.t1], tablas(b)[j.t2], j.clave)}

%-----------------------
%   ax: modificaciones
%-----------------------

\tadAxioma{tsMasModificadas(b)}{
	masModificadas(claves(tablas(b)),tablas(b), \\maxCambios(tablas(b)))
}

\tadAxioma{masModificadas(cs,ts,m)}{
	\IF $\emptyset$?(cs) THEN
		ts
	ELSE{ \IF \#cambios(ts[dameUno(cs)]) < m THEN
			borrar(dameUno(cs), masModificadas(sinUno(cs), ts, m) )
		ELSE
			masModificadas(sinUno(cs),ts,m)
		FI
		}
	FI		
}

\tadAxioma{maxCambios(ts)}{
	\IF vacio?(claves(ts)) THEN
		0
	ELSE
        max(\#cambios(ts[primc(ts)]),maxCambios(sinc(ts)))
    FI
}

\end{tad}



\end{document}
